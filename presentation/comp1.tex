\section[Component 1]{Component 1: HelloGeoWorld}
\subsection[Geographical Information System]{Geographical Information System}

%\Fontvi
\begin{frame}{Geospatial file formats}
	%\Fontvi
	\begin{beamerboxesrounded}{Vector file formate}
		\begin{itemize}
			\item Shape file, a binary 
			\item KML, XML notation 
			\item Geojson, json notation
		\end{itemize}
	\end{beamerboxesrounded}
	\begin{beamerboxesrounded}{Rastor file formate}
		\begin{itemize}
			\item tiff
			\item ascii  
		\end{itemize}
	\end{beamerboxesrounded}
\end{frame}

\begin{frame}{Quantum GIS- QGIS}
		%\Fontvi
		\begin{beamerboxesrounded}{}
			\begin{itemize}
				\item QGIS, cross-platform free and open-source desktop geographic information system (GIS) application
				\item Supports viewing, editing and analysis of geospatial data 
				\item Extensive support on different geospatial data types
				\item Written in C++ with legacy geospatial programs such as GEOS and SQLite. GDAL, GRASS GIS, PostGIS, and PostgreSQL
				\item Plugins as extension 
			\end{itemize}
		\end{beamerboxesrounded}
	\end{frame}
	
	
\begin{frame}{Google earth}
	%\Fontvi
	\begin{beamerboxesrounded}{}
		\begin{itemize}
			\item 3D representation of earth 
			\item Collection of  satellite imagery, aerial photography and GIS data onto a 3D globe
			\item 3D by DEM (Digital Elevation Models)
			\item Origin from Keyhole Earth Viewer
		\end{itemize}
	\end{beamerboxesrounded}
\end{frame}




\subsection[Jupyter notebook]{Jupyter notebook}
\begin{frame}{What are ipynb notebooks}
	%\Fontvi
	\begin{beamerboxesrounded}{}
		\begin{itemize}
			\item A command shell for interactive computing
			\item A browser-based notebook with support for code, text, mathematical expressions, inline plots and other media.
			\item For  introspection, rich media, shell syntax, tab completion, and history.
			\item Language agnostic 
			\item Enhances interactive data visualization and use of GUI toolkits.
			\item Easily shareable 
			\item enables parallel computing 
		\end{itemize}
	\end{beamerboxesrounded}
\end{frame}

\begin{frame}{Markdown language}
	%\Fontvi
	\begin{beamerboxesrounded}{}
		\begin{itemize}
			\item Text component of notebooks
			\item Lightweight markup language
			\item Defacto writing language in web
			\item Simple syntax
		\end{itemize}
	\end{beamerboxesrounded}
\end{frame}


\subsection[Do It Yourself 1]{Do It Yourself 1}
\begin{frame}{DIY 1}
	%\Fontvi
	\begin{beamerboxesrounded}{}
		\begin{itemize}
			\item Install QGIS, Google earth, ice break with geospatial data
			\item Install Anaconda python distribution 
			\item Install the required Jupyter notebook extension
			\item Tasks on Data frame
			\item Tasks on data visualization
		\end{itemize}
	\end{beamerboxesrounded}
\end{frame}

